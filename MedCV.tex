\documentclass[a4paper,11pt]{article}
%\documentclass[a4paper,10pt]{scrartcl}
\usepackage[margin=1in,footskip=0.25in]{geometry}
\usepackage[utf8]{inputenc}
\usepackage{enumitem}
\usepackage{outline}
\usepackage{longtable}
\usepackage{multicol}
\usepackage{tabularx}
\usepackage{pslatex}
\usepackage{fancyhdr}
\usepackage{multirow}

\title{\emph{Curriculum Vitae}}
\date{}

\pdfinfo{%
  /Title    ()
  /Author   ()
  /Creator  ()
  /Producer ()
  /Subject  ()
  /Keywords ()
}

\fancypagestyle{firstpage}{
\fancyhf{}
\renewcommand{\headrulewidth}{3pt}
\chead{\LARGE \itshape \bfseries Curriculum Vitae}
}

\fancypagestyle{otherpages}{
 \fancyhf{}
 \renewcommand{\headrulewidth}{1pt}
 \renewcommand{\footrulewidth}{1pt}
 \lhead{\itshape Curriculum vitae}
 \rhead{\itshape William Dampier, Ph.D}
 \cfoot{\thepage}
}

\makeatletter
\newenvironment{fullwidth}
    {\par
     \setlength{\@totalleftmargin}{0pt}%
     \setlength{\linewidth}{\hsize}%
     \list{}{\setlength{\leftmargin}{0pt}}
     \item\relax}
    {\endlist}
\makeatother

\renewcommand{\theenumi}{\textbf{\LARGE \Alph{enumi}}}



\begin{document}
\pagestyle{firstpage}

\begin{enumerate}
 \item {\LARGE \itshape \bfseries Name} \newline
 William Dampier
 \item {\LARGE \itshape \bfseries Home and Professional Mailing Address}
 \begin{itemize}
  \item[] {\Large \bfseries \itshape Home Address}
  \begin{itemize}
   \item[] 226 W. Rittenhouse Sq, Apt 2213
   \item[] Philadelphia, PA, 19103
   \item[] Cell Phone Number: (267) 403-0049
  \end{itemize}
  \item[] {\Large \bfseries \itshape Professional Address}
  \begin{itemize}
   \item[] Drexel University College of Medicine
   \item[] Department of Microbiology and Immunology
   \item[] 18th Floor New College Building
   \item[] Rm. 18105
   \item[] 245 N. 15th Street
   \item[] Philadelphia, PA 19102
   \item[] (215) 762-7340
   \item[] Electronic Mail:  wnd22@drexel.edu
  \end{itemize}
 \end{itemize}
 \item {\LARGE \itshape \bfseries Education}
 \begin{longtable}{p{0.15\textwidth}p{0.01\textwidth}p{0.65\textwidth}}
  2001-2006 & & Drexel University, Philadelphia, Pennsylvania \newline Major: Bioinformatics \newline Awarded Bachelor of Science degree \\
  2006-2010 & & Drexel University, Philadelphia, Pennsylvania \newline School of Biomedical Engineering and Health Sciences \newline Awarded Doctoral Degree \\
 \end{longtable}
 \item {\LARGE \itshape \bfseries Postgraduate Training}
 \begin{longtable}{p{0.15\textwidth}p{0.01\textwidth}p{0.65\textwidth}}
  2001-2006 & & Postdoctoral Researcher \newline Department of Computational Biology \newline GlaxoSmithKline Collegeville, Pennsylvania \newline Under the direction of Dr. Jim Brown, Director \\
 \end{longtable}
 \item {\LARGE \itshape \bfseries Employment History}
  \begin{longtable}{p{0.15\textwidth}p{0.01\textwidth}p{0.65\textwidth}}
  2002-2003 & & Research Assistant \newline Viropharma Inc. \newline Chester Springs, Pennsylvania \newline Research Interest: Small molecule inhibitors of Hep-C Virus\\
  \\
  2003-2004 & & Research Assistant \newline NovaFlora Inc. \newline Philadelphia, Pennsylvania \newline Research Interest: Genetic engineering of ornamental flowers\\
  \\
  2004 & & Research Assistant \newline GlaxoSmithKline \newline Upper Providence, Pennsylvania \newline Research Interest: Alzheimer’s research in \emph{C. elegans}\\
  \\
  2005-2006 & & Research Assistant \newline University of Pennsylvania \newline Department of Microbiology \newline Philadelphia, Pennsylvania \newline Research Interest: Entry methods of L. monocytogenes\\
  \\
  2008-2013 & & Assistant Director \newline Center for Integrated Bioinformatics \newline Drexel University, School of Biomedical Engineering \newline Philadelphia, Pennsylvania\\
  \\
  2011-2013 & & Research Assistant Professor \newline Drexel University \newline School of Biomedical Engineering \newline Philadelphia, Pennsylvania\\
  \\
  2013-present & & Research Assistant Professor \newline Department of Microbiology and Immunology \newline Drexel University College of Medicine \newline Philadelphia, Pennsylvania\\
 \end{longtable}
\pagestyle{otherpages}
 \item {\LARGE \itshape \bfseries Certification and Licensure} \newline
 N/A
 \item {\LARGE \itshape \bfseries Military Service} \newline
 N/A
 \item {\LARGE \itshape \bfseries Honors and Awards}
 \begin{longtable}{p{0.15\textwidth}p{0.01\textwidth}p{0.65\textwidth}}
  2009 & & 6th place in the Matlab Programming Contest. \newline More then 6000 entries world-wide\\
  \\
  2007 & & 2nd Place, Most Innovative Technology Poster \newline School of Biomedical Engineering and Health Systems, Drexel University \newline Philadelphia, Pennsylvania\\
  \end{longtable}
 \item {\LARGE \itshape \bfseries Memberships in Professional Societies}
  \begin{longtable}{p{0.15\textwidth}p{0.01\textwidth}p{0.65\textwidth}}
  2013-present & & The International Society of NeuroVirology\\
  \end{longtable}
 \item {\LARGE \itshape \bfseries Professional Committees and Administrative Service}
 \begin{itemize}
  \item[] {\Large \bfseries \itshape Institutional Service}
  \begin{longtable}{p{0.15\textwidth}p{0.01\textwidth}p{0.65\textwidth}}
  2010-present & & Senior Design Committee Member \newline Drexel University, School of Biomedical Engineering \newline Philadelphia, Pennsylvania \\
  \\
  2012-present & & Senior Design Committee Advisor \newline Drexel University, School of Biomedical Engineering \newline Philadelphia, Pennsylvania \\
  \end{longtable}
  \item[] {\Large \bfseries \itshape Extramural Service}
  \begin{longtable}{p{0.15\textwidth}p{0.01\textwidth}p{0.65\textwidth}}
  2011-present & & Python Core Contributor \\
  2011-present & & Python Scipy Contributor \\
  2011-present & & Python Django Contributor \\
  2011-present & & Python Scikits-Learn Contributor \\
  \end{longtable}
  \item[] {\Large \bfseries \itshape Journal Editorial and Review Responsibilities}
  \begin{longtable}{p{0.15\textwidth}p{0.01\textwidth}p{0.65\textwidth}}
  2010-present & & BMC Bioinformatics (Reviewer) \\
  2011-present & & BMC Medical Genomics (Reviewer) \\
  2012-present & & International Journal of Genomics (Reviewer) \\
  \end{longtable}
 \end{itemize}
 \item {\LARGE \itshape \bfseries Community Service} \newline
 N/A
 \item {\LARGE \itshape \bfseries Educational Activities}
 \begin{itemize}
  \item[] {\Large \bfseries \itshape Teaching Experience}
   \begin{longtable}{p{0.15\textwidth}p{0.01\textwidth}p{0.65\textwidth}}
  2008-present & & Head Instructor \newline Drexel Judo Club (10 hours/week) \newline Drexel University \newline Philadelphia, Pennsylvania \\
  \\
  2010-2012 & & Head Instructor \newline UPenn Judo Club (4 hours/week) \newline University of Pennsylvania \newline Philadelphia, Pennsylvania \\
  \\
  2010-2011 & & Adjunct Professor, (BMES 505-507) \newline Math for Biomedical Scientist (3 credits each) \newline Drexel University School of Biomedical Engineering \newline Philadelphia, Pennsylvania \\
  \\
  2010-present & & Adjunct Professor, (BMES 375) \newline Computational Biology (4.5 credits) \newline Drexel University School of Biomedical Engineering \newline Philadelphia, Pennsylvania \\
  \end{longtable}
 \end{itemize}
 \item {\LARGE \itshape \bfseries Clinical Activities} \newline
 N/A
 \item {\LARGE \itshape \bfseries Grant Support}
 \begin{itemize}
  \item[] {\Large \bfseries \itshape Pending} (Ordered by submission)
  
  
  \begin{longtable}{lr}
    \textbf{R01 RFA-MH-14-170 (Julio)} & 4/01/2014 - 3/31/2019 \\
    National Institutes of Health & Direct Cost - \$2,492,003.45  \\
    \multicolumn{2}{p{0.973\textwidth}}{\bfseries The HIV-1 CNS reservoir: macrophage tropism, latency and eradication.} \\
    Role on Project:  Co-Investigator & Salary Coverage: 5\% Effort \\
    Submitted: 9/17/2013\\
    \end{longtable}
    \begin{fullwidth}
     HIV-1 envelope glycoproteins (Env) present in the brain of infected individuals are characterized by their macrophage tropism, or the ability to efficiently infect macrophages. Indeed, macrophages and microglia, the resident brain macrophage, support most if not all productive viral replication within the brain compartment. HIV-1 is believed to enter the CNS early after systemic infection and might either be cleared, or continue to reside there and perhaps maintain a low level, continuous replication, which may lead, at least in a subset of infected individuals, to the development of neurocognitive manifestations of varying severity, known collectively as HIV-1-associated neurocognitive disorders. The brain thus may contain productively and latently infected cells, and in the context of combination anti-retroviral therapy (cART) and eradication efforts, may become a viral reservoir. However, it remains to be demonstrated whether HIV-1 infection within the brain can eventually re-seed non-CNS tissues after viral eradication using cART. In other words, whether the CNS can truly act as a functional reservoir for HIV-1. Macrophage tropic Env are also known to have altered reactivity to certain types of neutralizing antibodies (nAbs) and small molecules/inhibitors that target various steps of the receptor-mediated binding and entry process. We and others have shown that macrophage-tropic Env have increased exposure of epitopes that are recognized by Abs that are generally produced in HIV-1 infected individuals, resulting in increased sensitivity to neutralization by HIV-1 positive sera, and by Abs targeting the CD4-induced (CD4i) epitopes that partially overlap with the bridging sheet regions that form the co-receptor binding site. In addition, an inverse relationship between macrophage tropism and sensitivity to inhibition by certain gp120- or gp41-targeted entry or fusion inhibitors, respectively (such as BMS-378806, T20 and T1249) has also been observed, whereas another class of gp120 inhibitors, the 12p1-derived peptides, actually seems to target more efficiently macrophage-tropic than non-macrophage-tropic Env. Thus, it is of great interest to determine whether macrophages and microglia infected with viruses containing macrophage-tropic Env will only be producing a progeny of virions containing Envs with this phenotype or rather a mixture of viruses containing both macrophage-tropic and non-macrophage-tropic Env. Our overall hypothesis is that macrophages/microglia infected with viruses with macrophage-tropic Env produce progeny virions with both macrophage- and non-macrophage-tropic Env, which will be able to re-seed infection of non-CNS peripheral tissues after viral eradication with cART. To test this hypothesis, or specific goals are: (i) to investigate in vitro how HIV-1 is able to establish and maintain viral replication in the CNS, and how the CNS can potentially serve as a reservoir from which the virus can re-seed peripheral tissues in the context of viral eradication efforts; (ii) to finely define the determinants for differential sensitivity of macrophage-tropic and non-macrophage-tropic Env to two types of entry inhibitors (BMS-378806 and related small molecules, and 12p1-related peptides); and (iii) to use the RAG-hu mouse model of HIV-1 infection to define in vivo early mechanisms of HIV-1 neuroinvasion and establishment of viral replication and/or latency in the brain, to determine the potential efficacy of targeted interventions against HIV-1 infection of the CNS and re-infection of peripheral tissues, and to study the role of cellular trafficking and CNS inflammation in maintaining viral persistence within this compartment.
    \end{fullwidth}
  
    \begin{longtable}{lr}
    \textbf{R01 PA-11-260 (Wigdahl)} & 4/01/2014 - 3/31/2019 \\
    National Institutes of Health & Direct Cost - \$2,424,486.75 \\
    \multicolumn{2}{p{0.973\textwidth}}{\bfseries Defining the HIV-1 R5 genotype beyond the envelope and in viral reservoirs} \\
    Role on Project:  Co-Investigator & Salary Coverage: 50\% Effort \\
    Submitted: 9/9/2013\\
    \end{longtable}
    \begin{fullwidth}
     HIV-1 is known to interact with the cell surface receptor CD4 in conjunction with a chemokine coreceptor, either CCR5 (R5) or CXCR4 (X4). R5-utilizing viruses are more commonly found emerging in the peripheral blood (PB) following the selection bottleneck after sexual transmission, in the early asymptomatic stages of disease and treatment-naïve patients, within the brain at end-stage disease, and in cerebrospinal fluid (CSF) during HIV CNS disease; whereas X4-utilizing viruses emerge later in the course of disease in approximately 50-60\% of infected individuals with an average time to emergence of 5 years. The presence of X4 virus in an infected patient has been shown to be a predictor of lower CD4+ T-cell count, higher viral load, and greater degree of HIV disease severity. However, more recent analyses have implied that lower CD4+ T-cell counts and higher viral loads may be observed with dual/mixed tropic virus compared to that associated with patients carrying R5 or X4 virus as their predominant virus. In addition, we have recently demonstrated through bioinformatics analyses that there appears to be defined subgroups of R5 viruses that may have specific pathogenic properties and may be more or less prevalent at different stages of disease, within different cellular/tissue compartments involved in immunologic and neurologic dysfunction, and development of viral reservoirs prior to or after the initiation of antiretroviral therapy.  Furthermore, recent studies have also indicated that X4- and R5-specific colinear nucleotide and amino acid sequences as defined by the position-specific scoring matrix (PSSM) algorithm utilizing the Env-V3 sequence exist within the HIV-1 long terminal repeat (LTR) as well as Tat and Vpr, respectively. These highly innovative concepts have paved the way for defining in greater detail the molecular architecture of the X4, R5, and R5 subgroups (and eventually the dual tropic virus) beyond the well characterized signature sequences contained within Env-V3 sequence to the rest of the viral genome, in particular the LTR, Tat, and Vpr sequences, the focus of this proposal. The studies outlined in this proposal will utilize PB and CSF samples from cross-population and longitudinal studies from several cohorts in addition to brain samples collected from the National NeuroAIDS Tissue Consortium (NNTC) for sequencing and structure/function analyses to examine the working hypothesis that R5 subgroups can be categorized by sequence clustering and the individual subgroups correlate with disease severity and other clinical parameters including neurocognitive impairment and immune activation and relates to reservoir development and cellular compartmentalization.  The specific aims of this application are to: (1) Enhance collection and definition of samples from patients classified as nonusers with respect to drugs of abuse utilizing NGS; (2) Determine envelope phenotype for patient samples and categorize into X4-utilizing or R5-utilizing with classification into R5 subgroups; (3) Identify the immune activation profile of various R5 subgroups and correlate with clinical parameters including neurocognitive impairment; and (4) Determine the function of the viral proteins beyond the envelope and correlate the function to the various R5 subgroups.
    \end{fullwidth}
    
     \begin{longtable}{lr}
    \textbf{R01 PA-11-260 (Dampier)} & TBD \\
    Burrooughs-Welcome & Direct Cost - \$500,000 \\
    \multicolumn{2}{p{0.973\textwidth}}{\bfseries Bioinformatics analysis of HIV-1 neurocognitive decline} \\
    Role on Project:  Principal Investigator & Salary Coverage: TBD \\
    Submitted: 9/9/2013\\
    \end{longtable}
    \begin{fullwidth}
     With the continued introduction of new therapeutic interventions and the existence of more than 30 FDA-approved therapeutic agents to control HIV-1 infection, experimental attention has been shifting to the long-term management of chronic and progressive infection in an aging HIV-1-infected patient population. The CDC estimates that by 2020 half of the HIV population will be over age 50; we have already begun to see an increased frequency of milder forms of HIV-associated neurocognitive disorders (HAND) in infected adult and aging individuals. This neurological decline is further accelerated by drug abuse endemic in urban populations which, in Center City Philadelphia, is dominated by cocaine and cannabinoid use and complicated by poly-drug use and other comorbidities. In the DrexelMed (DM) HIV/AIDS Genetic Analysis Cohort (PI:Dr. Brian Wigdahl) more than 500 patients have been enrolled and are followed at 6-month intervals with >1,500 visits recorded. At each visit a minibedside neurocognitive exam is administered with neurophyschologic follow-up evaluation along with blood draws for drug-testing, cytokine measurements, CD4/CD8 counts, viral load determinations, and a number of other clinical measurements along with an assessment of viral genetic variation, which occurs despite antiretroviral therapy. The hypothesis of this proposal is that HAND progression is a complex interplay of viral genetic factors and host factors such as drug abuse, immune reaction, therapeutic intervention and associated comorbidities that unfolds over many years. 
    \end{fullwidth}

    
    \begin{longtable}{lr}
    \textbf{U01 PAR-13-029 (Nonnemacher)} & 10/01/2014 - 9/30/2017 \\
    National Institutes of Health & Direct Cost - \$199,967.64 \\
    \multicolumn{2}{p{0.973\textwidth}}{\bfseries  CSF HIV compartmentalization and neuroimmune correlates of HAND in military HIV+ } \\
    Role on Project:  Co-Investigator & Salary Coverage: 15\% Effort \\
    Submitted: 3/20/2013 \\
    \end{longtable}
    \begin{fullwidth}
    Nearly one fifth to one third of HIV-infected individuals develop neurocognitive deficits, collectively referred to as HIV associated neurocognitive disorders (HAND), despite adequate combination antiretroviral therapy (cART) and excellent virological control in blood. HIV is detected in the cerebrospinal fluid (CSF) soon after a primary infection  and traffics into brain early in the course of infection where it resides in perivascular macrophages and microglial cells, sites of productive replication and viral evolution. Importantly the virus  enters the brain in only about 50\% of patients as detected by molecular methods at the time of death. For those with central nervous system (CNS) infection, complete suppression of HIV replication in the brain is challenging due to the selectively permeable blood brain barrier that interferes with bioavailability of cART in brain.  Multiple mechanisms have been proposed as to how low level viral replication may lead to neurocognitive disorders.  Thus it is critically important to identify patients who may not have HIV in the brain as they may have the lowest risk of HAND and/or the best chance for curative therapies.  A cure for HIV infection is not possible unless safe heavens of the virus are purged and total eradication of HIV from the host is achieved.  An essential step toward this end is identification of host and viral responses in the CNS both among recently seroconverted individuals and those on long term cART.  In the U.S. military, although HIV infected members may remain on active duty, they are restricted by policy from some positions (e.g. pilot of high performance aircraft) and for the most part from overseas deployments, largely due to concerns about neurocognitive dysfunction.  This proposal brings together investigators in the fields of HIV neuroimmunology, neurovirology and neuroradiology with a team of clinical researchers leveraging a unique population of military members who are routinely screened for HIV infection.  We will employ highly sensitive biomarker assays as well as viral and proviral HIV deep sequencing and evolutionary analysis to characterize HIV infections in the CNS of these subjects in detail and longitudinally.
  \end{fullwidth}
  
  \begin{longtable}{lr}
    \textbf{U01 PAR-12-222 (Wigdahl)} & 7/01/2013 - 6/30/2018 \\
    National Institute on Drug Abuse & Direct Cost - \$2,498,831.50 \\
    \multicolumn{2}{p{0.973\textwidth}}{\bfseries HIV-induced cognitive \& immune impairment in a preferential cocaine using cohort } \\
    Role on Project:  Co-Investigator & Salary Coverage: 25\% Effort\\
    Submitted: 12/11/2012\\
    \end{longtable}
   \begin{fullwidth}
   HIV-1 infection and subsequent therapeutic management can be modulated by a variety of cofactors, including substance abuse, which continues to be a major driver of global infection due to classical risk factors including risky sexual behavior and injection drug use (IDU).  However, HIV prevalence among non-injecting drug users has also grown to levels similar to those observed for IDU, and polysubstance use is common among HIV-1-infected populations.  Through development of the HIV-1-infected DREXELMED HIV/AIDS Genetic Analysis Cohort, we have identified cocaine abuse as the most prevalent substance of abuse in this Philadelphia-centered patient population with preferential cocaine use clearly evident in the population paralleled by a substantial cohort of polysubstance abusers as determined by routine drug screening.  Cocaine has been shown to affect both immune activation and cellular gene expression including transcription factors resulting in impact on HIV-1 gene expression and pathogenesis.  HIV-1 also displays extensive sequence variation induced by many host-specific and comorbidity pressures, including substance abuse (most notably cocaine), while simultaneously maintaining functions that are critical to replication and infectivity. In this proposal, we will utilize molecular and systems biology approaches to generate an overall picture of important pathways in HIV-1 pathogenesis and how cocaine interacts with and affects these pathways and the development of viral quasispecies over the course of disease.  The overall Hypothesis of the proposed studies is that alterations in the cross talk between HIV-1 and the host due to cocaine abuse will increase the rate of HIV-1 disease progression and result in immune dysregulation with the genesis of a viral quasispecies unique to preferential users of cocaine.  We have illustrated the importance of linear protein sequence motifs and LTR footprints in viral-host cross talk during the course of our preliminary investigations.  In this application, we will quantify the HIV-1 protein sequence motifs and TF footprints on the HIV-1 LTR enriched in sequences associated with cocaine abuse as well as quantify and compare global gene and immune activation profiles of HIV-1-infected individuals with no history of substance abuse (PN) and with preferential cocaine addiction (PC). The Specific Aims of this proposal are to (1) examine the impact of cocaine on HIV-1 LTR and viral protein host motifs; (2) enhance the clinical demographics and neuropsychological analysis associated with PN, PC, and multi-use subcohorts within the DREXELMED HIV/AIDS Genetic Analysis Cohort; and (3) determine the impact of preferential cocaine use on immune regulatory pathways in HIV-1-infected patients.  This proposal will develop and utilize unique HIV-1-infected patient subcohorts to determine the overall impact of cocaine use on human and viral gene expression, immune activation, and selection of viral quasispecies.  These studies will significantly impact research on HIV/AIDS and cocaine abuse through the identification of new targets and strategies for anti-HIV-1 therapeutics and strategies to minimize the impact of substance abuse on HIV disease progression and overall human health.
   \end{fullwidth}

   \begin{longtable}{lr}
    \textbf{R21 PAR-12-174 (Pirrone)} & 4/1/2013 - 3/31/2015 \\
    National Institutes of Health & Direct Cost - \$275,000.00 \\
    \multicolumn{2}{p{0.973\textwidth}}{\bfseries HIV/HCV coinfection alters immune cell function in an aging population } \\
    Role on Project:  Co-Investigator & Salary Coverage: 10\% Effort\\
    Submitted: 8/7/2012\\
    \end{longtable}
   \begin{fullwidth}
    A large number of HIV-1-infected individuals are coinfected with hepatitis C virus (HCV).  Both HIV-1 and HCV share similar modes of transmission, and as such approximately 30\% of HIV-1-infected patients are coinfected with HCV, suggesting that there are greater than 10 million coinfected individuals globally.  Studies have demonstrated that the normal pathogenesis of HCV infection is significantly impacted by infection with HIV-1, with coinfected individuals possessing higher HCV viral loads, and more rapid progression to serious liver disease, including cirrhosis and hepatocellular carcinoma.  Additionally, it has also been demonstrated that the efficacy of HCV therapy is decreased in HIV/HCV coinfected individuals.  The advent of highly active retroviral therapy (HAART) has resulted in significantly lower mortality rates, resulting in an increasingly larger number of affected older adults. It has previously been shown that HIV-positive patients 50 years and older have a 2 to 3 times greater frequency of dementia than that observed in HIV-positive patients between the ages of 20 and 39. Immune failure also is a common occurrence within both aging and HIV-infected population, suggesting a possible synergistic effect of aging and HIV infection on accelerated immune activation and possible immunosenescence. In a similar manner to HIV monoinfected patients, HIV/HCV coinfected patients are also increasing in age as therapies are getting more successful at extending the lifespan of infected individuals.  However, mortality associated with liver disease still remains high in HIV-infected patients.  In addition to direct liver effects caused by both HIV and HCV, coinfected patients are also at increased risk of cardiovascular disease, stroke, diabetes mellitus and other comorbidities are typically associated with aging.  Given this the hypothesis of the proposed studies is that HIV/HCV coinfection synergistically enhances the effects of age by accelerating immune system activation. In this proposal, we will utilize molecular and cellular biology approaches to generate an overall picture of important pathways in HIV/HCV pathogenesis and to determine how age affects these pathways, and the potential relevance of coinfection to aging.  We will quantify and compare immune activation profiles with gene expression profiles of adult and aged patients with respect to HIV/HCV coinfection to ascertain the effects that coinfection has with regard to chronic immune activation within aged patients.  The Specific Aims of this proposal are to (1) Identify the effect of age on the immune regulatory molecules as well as human genes/proteins/pathways in coinfected HIV/HCV patients versus HIV-1 monoinfected patients and (2) Determine the functionality of immune cell populations and mechanisms of action involved in chronic immune activation and viral immunity in adult and aged HIV/HCV coinfected patients.  This proposal will develop and utilize unique HIV/HCV coinfected patient cohorts to determine the impact of age on human gene expression, immune activation, and development of neurocognitive impairment.  These studies will impact research on HIV/HCV/Age by identifying new targets and strategies for anti-HIV-1 therapeutics and by minimizing the impact of both age and HCV on HIV-1 disease.
   \end{fullwidth}

 
  
  \item[] {\Large \bfseries \itshape Past}
  \begin{longtable}{lr}
     & 8/15/2012 - 12/31/2012 \\
    Thomas Jefferson University & Direct Cost - \$34,556.61  \\
    \multicolumn{2}{p{0.973\textwidth}}{\bfseries Jefferson Kimmel Center - Drexel Bioinformatics Research Alliance.} \\
    Role on Project:  Co-Investigator & Salary Coverage: 50\% Effort  \\
  \end{longtable}
  \begin{longtable}{lr}
     & 8/15/2012 - 12/31/2012 \\
    GlaxoSmithKline & Direct Cost - \$32,164.01  \\
    \multicolumn{2}{p{0.973\textwidth}}{\bfseries GSK - Drexel University Bioinformatics Research Collaboration.} \\
    Role on Project:  Co-Investigator & Salary Coverage: 50\% Effort  \\
  \end{longtable}
  
 \end{itemize}
 

 \item {\LARGE \itshape \bfseries Graduate Students, Postdoctoral Fellows, and Postgraduate Medical Trainies}
 \begin{itemize}
  \item[] {\Large \bfseries \itshape Supervision of Undergraduate Student Research}
 
   \begin{longtable}{p{0.15\textwidth}p{0.01\textwidth}p{0.65\textwidth}}
  2012-present & & Alycia Logue, School of Biomedical Engineering, Drexel University, Philadelphia, Pennsylvania\\
  2012-present & & Lilly Hippel, School of Biomedical Engineering, Drexel University, Philadelphia, Pennsylvania\\
  2012-present & & Mena Schiano, School of Biomedical Engineering, Drexel University, Philadelphia, Pennsylvania\\
  2012-present & & Mike Ryan, School of Biomedical Engineering, Drexel University, Philadelphia, Pennsylvania\\
  \end{longtable}

  \item[] {\Large \bfseries \itshape Co-Supervision of Graduate Student Research}
    \begin{longtable}{p{0.15\textwidth}p{0.01\textwidth}p{0.65\textwidth}}
  2012-present & & Gregory Antell, School of Biomedical Engineering Graduate Program (Ph.D degree student) Drexel University, Philadelphia, Pennsylvania
  \end{longtable}
 \end{itemize}

 \item {\LARGE \itshape \bfseries Bibliography}
 \begin{itemize}
  \item[] {\Large \bfseries \itshape Published full-length papers}
  \begin{enumerate}[label=\arabic{enumii}.]
   \item Gormley M., \textbf{Dampier W.}, Ertel A., Karacali B., Tozeren A. Prediction potential of candidate biomarker sets identified and validated on gene expression data from multiple datasets. BMC Bioinformatics, Oct. 2007; 8:415 Cited by 17
   \item \textbf{Dampier W.}, Tozeren A. Signaling perturbations induced by invading H. pylori proteins in the host epithelial cells: A mathematical modeling approach. Journal of Theoretical Biology, Sept. 2007; 248(1):130 Cited by 8
   \item Layton B., D'Souza A., \textbf{Dampier W.}, Zeiger A., Sabur A., Jean-Charles J. Collagen's triglycine repeat number and phylogeny suggest an interdomain transfer event from a Devonian or Silurian organism into Trichodesmium erythraeum. J Mol Evol. June 2008; 66(6):539. Cited by 7
   \item Evans P., \textbf{Dampier W.}, Ungar L., Tozeren A. Prediction of HIV-1 virus-host protein interactions using virus and host motifs. BMC Med Genomics, May 2009; 2:27 Cited by 34, Highly Accessed (As determined by Biomed Central)
   \item \textbf{Dampier W.}, Evans P., Ungar L., Tozeren A. Host sequence motifs shared by HIV-1 predict patient response to antiretroviral therapy. BMC Med Genomics, July 2009; 2:47 Cited by 13
   \item Zhou J., Wang C., Wang Z., \textbf{Dampier W.}, Wu K., Casimiro M., Chepelev L., Popov V., Quong A., Tozeren A., Zhao K., Lisanti M., Pestell R. Attenuation of Forkhead Signaling by the Retinal Determination Factor DACH1. Proceedings of the National Academy of Sciences, March 2010 Cited by 13
   \item Dawany N., \textbf{Dampier W.}, Tozeren A. Large-scale integration of microarray data reveals genes and pathways common to multiple cancer types. Int J Cancer. Dec 2010. Cited by 12
   \item Sarmady M., \textbf{Dampier W.}, Tozeren A. HIV Protein Sequence Hotspots for Crosstalk with Host Hub Proteins PLOS One. June 2011, Cited by 5
   \item Sarmady M., \textbf{Dampier W.}, Tozeren A. Sequence- and Interactome- Based Prediction of Viral Protein Hotspots Targeting Host Proteins: A Case Study for HIV Nef. PLOS One. June 2011, Cited by 2
   \item Casimiro MC., Crosariol M., Loro E., Ertel A., Yu Z., \textbf{Dampier W.}, Saria EA., Pestell R. ChIP sequencing of cyclin D1 reveals a transcriptional role in chromosomal instability in mice, The Journal of Clinical Investigation 122 (3), 833, March 2011, Cited by 16
   \item Smith SB., \textbf{Dampier W.}, Tozeren A., Brown JR., Magid-Slav M. Identification of Common Biological Pathways and Drug Targets Across Multiple Respiratory Viruses Based on Human Host Gene Expression Analysis. PloS one 7 (3), e33174. March 2011, Cited by 14
   \item Clark PM., Dawany N., \textbf{Dampier W.}, Byers SW., Pestell RG., Tozeren A. Bioinformatics analysis reveals transcriptome and microRNA signatures and drug repositioning targets for IBD and other autoimmune diseases. Inflammatory Bowel Diseases, June 2012, Cited by 5
  \end{enumerate}
  \item[] {\Large \bfseries \itshape Abstracts}
  \begin{enumerate}[label=\arabic{enumii}.]
   \item Aiamkitsumrit B., Nonnemacher M., Pirrone V., Zhong W., Frantz B., Rimbey M., Passic S., Blakey B., Parikh N., Martin-Garcia J., Downie D., Lewis S., Jacobson J., Moldover B., \textbf{Dampier W.}, Wigdahl B. Identification of HIV-1 X4, R5, and R5 subgroup genetic signatures in the viral promoter, Tat, and Vpr. University of Pennsylvania CFAR 11th Annual Research Retreat, Philadelphia, PA, December 3, 2012
   \item \textbf{Dampier W.}, Nonnemacher M.,, Pirrone V., Williams J., Aiamkitsumrit B., Wojno A., Passic S., Blakey B., Zhong W., Moldover B., Feng R., Downie D., Lewis S., Jacobson J., Wigdahl B.  Impact of substance abuse on HIV-1 LTR single nucleotide polymorphisms (SNPs) and disease progression in a clinical cohort.  Society for Neuroimmune Pharmacology 18th Scientific Conference, San Juan, Puerto Rico, April 3-6, 2013.
   \item Nonnemacher M., Pirrone V., \textbf{Dampier W.}, Aiamkitsumrit B., Williams J., Shah S., Wojno A., Passic S., Blakey B., Zhong W., Moldover B., Feng R., Downie D., Lewis S., Jacobson J., Wigdahl B.  HIV-1 LTR single nucleotide polymorphisms (SNPs) correlate with clinical disease parameters. Society for Neuroimmune Pharmacology 18th Scientific Conference, San Juan, Puerto Rico, April 3-6, 2013.
   \item Antell G., Nonnemacher M., Pirrone V., \textbf{Dampier W.}, Aiamkitsumrit B., Williams J., Shah S., Wojno A., Passic S., Blakey B., Zhong W., Moldover B., Feng R., Downie D., Lewis S., Jacobson J., Wigdahl B.  HIV-1 LTR single nucleotide polymorphisms (SNPs) that correlate with clinical disease parameters are found in both the peripheral blood and brain compartments.  Society for Neuroimmune Pharmacology 18th Scientific Conference, San Juan, Puerto Rico, April 3-6, 2013.
   \item Nonnemacher M., Strazza M., Pirrone V., Lin W., Feng R., \textbf{Dampier W.}, Wigdahl B.  Use of an in vitro model of the blood brain barrier to examine the effects of aging.  Translational Medicine \& Applied Biotechnology Workshop on Cognition and Aging, Drexel University College of Medicine, Philadelphia, PA, June 5, 2013.
   \item Pirrone V., Nonnemacher M., Passic S. R., Parikh N., Aiamkitsumrit B., \textbf{Dampier W.}, Katsikis P., Mueller Y., Sell C., Libon D., Moldover B., Feng R., Jacobson J. M., Wigdahl B.  Aging in the HIV-1-infected population: Impact on markers of HIV-1 disease.  Translational Medicine \& Applied Biotechnology Workshop on Cognition and Aging, Drexel University College of Medicine, Philadelphia, PA, June 5, 2013.
   \item Aiamkitsumrit B., Nonnemacher M., Pirrone V., Zhong W., Frantz B., Rimbey M., Passic S., Blakey B., Parikh N., Martin-Garcia J., Downie D., Lewis S., Jacobson J. M., Moldover B., \textbf{Dampier W.}, Wigdahl B. Identification of HIV-1 X4, R5, and R5 subgroup genetic signatures in the LTR, Tat, and Vpr.  2013 International Symposium on Molecular Medicine and Infectious Disease, Drexel University College of Medicine, Philadelphia, PA, June 17-21, 2013.
   \item Antell G., Nonnemacher M., Pirrone V., \textbf{Dampier W.}, Aiamkitsumrit B., Williams J., Shah S., Wojno A., Passic S., Blakey B., Zhong W., Moldover B., Feng R., Downie D., Lewis S., Jacobson J., Wigdahl B. HIV-1 LTR single nucleotide polymorphisms (SNPs) that correlate with clinical disease parameters are found in both the peripheral blood and brain.  2013 International Symposium on Molecular Medicine and Infectious Disease, Drexel University College of Medicine, Philadelphia, PA, June 17-21, 2013.
   \item Parikh N., \textbf{Dampier W.}, Feng R., Passic S., Zhong W., Frantz B., Aiamkitsumrit B., Pirrone V., Nonnemacher M., Jacobson J. M., Wigdahl B. Cocaine alters cytokine profiles within HIV-1-infected African American individuals in the DREXELMED HIV/AIDS Genetic Analysis Cohort 2013 International Symposium on Molecular Medicine and Infectious Disease, Drexel University College of Medicine, Philadelphia, PA, June 17-21, 2013.
   \item Williams J., \textbf{Dampier W.}, Nonnemacher M., Pirrone V., Aiamkitsumrit B., Wojno A., Passic S., Blakey B., Zhong W., Moldover B., Feng R., Downie D., Lewis S., Jacobson J. M., Wigdahl B. Impact of substance abuse on HIV-1 LTR single nucleotide polymorphisms (SNPs) and disease progression in a clinical cohort.  2013 International Symposium on Molecular Medicine and Infectious Disease, Drexel University College of Medicine, Philadelphia, PA, June 17-21, 2013.
   \item Aiamkitsumrit B., Nonnemacher M., Zhong W., Russo T., Pirrone V., Frantz B., Rimbey M., Passic S., Blakey B., Parikh N., Martin-Garcia J., Jacobson J., Moldover B., \textbf{Dampier W.}, Wigdahl B. Differential HIV-1 X4 and R5 genetic signatures within the LTR, Tat and Vpr. Journal of Neurovirology, Washington DC, October 25-30, 2013.
   \item \textbf{Dampier W.}, Parikh N., Nonnemache, M., Pirrone V., Williams J., Aiamkitsumrit B., Passic S., Zhong W., Moldover B., Feng R., Jacobson J., Wigdahl B. Longitudinal analysis of the impact of substance abuse on HIV-1-associated neurological decline in the DrexelMed HIV/AIDS Genetic Analysis Cohort. Journal of Neurovirology, Washington DC, October 25-30, 2013.
   \item Parikh N., \textbf{Dampier W.}, Feng R., Passic S., Zhong W., Aiamkitsumrit B., Pirrone V., Nonnemacher M., Jacobson J., Wigdahl B. Cocaine alters immunomodulatory profiles within HIV-1-infected African American individuals in the DREXELMED HIV/AIDS Genetic Analysis Cohort. Journal of Neurovirology, Washington DC, October 25-30, 2013.
   \item Zhong W., Pirrone V., Nonnemacher M., Parikh N., Aiamkitsumrit B., \textbf{Dampier W.}, Katsikis P., Mueller Y., Sell C., Libon D., Moldover B., Feng R., Jacobson J., Wigdahl B. Impact of Aging on markers of HIV-1 disease. Journal of Neurovirology, Washington DC, October 25-30, 2013.
   \item Williams J., \textbf{Dampier W.}, Nonnemacher M., Pirrone V., Aiamkitsumrit B., Wojno A., Passic S., Blakey B., Zhong W., Moldover B., Feng R., Downie D., Lewis S., Jacobson J., Wigdahl B. Use of drugs of abuse impact HIV-1 LTR single nucleotide polymorphisms (SNPs) in the DrexelMed HIV/AIDS Genetic Analysis Cohort. Journal of Neurovirology, Washington DC, October 25-30, 2013.
   \item Antell G., Nonnemacher M., Pirrone V., \textbf{Dampier W.}, Aiamkitsumrit B., Williams J., Shah S., Passic S., Blakey B., Zhong W., Moldover B., Feng R., Jacobson J., Wigdahl B. Multiple HIV-1 LTR single nucleotide polymorphisms (SNPs) that occur in peripheral blood and correlate with disease severity are also present in infected brain samples. Journal of Neurovirology, Washington DC, October 25-30, 2013.
   \item Pirrone V., Nonnemacher M., \textbf{Dampier W.}, Aiamkitsumrit B., Williams J., Shah S., Passic S., Blakey B., Zhong W., Moldover B., Feng R., Jacobson J., Wigdahl B. HIV-1 LTR single nucleotide polymorphisms (SNPs) correlate with clinical disease parameters. Journal of Neurovirology, Washington DC, October 25-30, 2013.
  \end{enumerate}

 \end{itemize}

 \item {\LARGE \itshape \bfseries Research Presentations}
 \begin{itemize}
  \item[] {\Large \bfseries \itshape Oral Presentations by Invitation}
   \begin{longtable}{p{0.15\textwidth}p{0.01\textwidth}p{0.65\textwidth}}
  2009 & & A Machine Learning Technique for the Classification of Therapeutic Interventions for HIV-1 Patients. \newline Villanova Computer Science Colloquium \newline Villanova, Pennsylvania \\
  \\
  2009 & & Classification of Therapeutic Response in HIV-1 Patients Using Functional Motifs. \newline GPBA Annual Research Retreat \newline Philadelphia, Pennsylvania \\
  \\
  2010 & & Co-Evolution in Viral Genomes. \newline GlaxoSmithKline Invited Lecture. \newline 	Philadelphia, Pennsylvania \\
  \\
  2011 & & Computational Analysis Pipelines in Python. \newline Invited Tutorial Children’s Hospital of Philadelphia \newline Philadelphia, Pennsylvania \\
  \end{longtable}
  \item[] {\Large \bfseries \itshape Invited Lectures}
   \begin{longtable}{p{0.15\textwidth}p{0.01\textwidth}p{0.65\textwidth}}
   2008 & & Multiple Alignments from a Bioinformatics Perspective. \newline Department of Electrical Engineering at Drexel University. \newline Invited lecture for Genomics Signals Processing ECE-690 \newline 	Philadelphia, Pennsylvania \\
   \\
   2008 & & An Overview of Molecular Evolution. \newline Department of Mechanical Engineering at Drexel University. \newline Invited lecture for MechanoEvolution MEM-380 \newline Philadelphia, Pennsylvania \\
   \\
   2010 & & Quantitative Methods for Analyzing Biological Reactions. \newline Invited Lecture, Izmir Institute of Technology \newline Izmir, Turkey \\
  \end{longtable}
 \end{itemize} 
\end{enumerate}
\pagestyle{otherpages}

\end{document}
